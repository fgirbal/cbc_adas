\begin{abstract}
Research into safety in autonomous and semi-autonomous vehicles has, so far, largely been focused on testing and validation through simulation. Due to the fact that failure of these autonomous systems is potentially life-endangering, formal methods arise as a complementary approach.

This thesis studies the application of formal methods to the verification of different profiles of human drivers (built using the cognitive architecture ACT-R), and to the design of correct-by-construction Advanced Driver Assistance Systems (ADAS) using strategy synthesis. The situation considered is a 2-lane highway scenario and the interactions that arise for various profiles of drivers (e.g. follow, crash or overtake another vehicle). 

% The human driver behaviour is modelled using the cognitive architecture ACT-R and abstracted using pre-simulation and model writing automation techniques. Afterwards, the abstracted human driver model (discrete-time Markov chain) is verified using safety and liveness metrics. Finally, the full system of the human and the ADAS is built through the augmentation of the original model to include all possible actions (transforming it into a Markov decision process), allowing the ADAS to be synthesised using multi-objective properties.

The results show that, in this situation, the synthesised ADAS improve both safety and efficiency of the overall system when compared to the human driver alone. 

The dissertation distinguishes itself from previous work done in this area due to the complex nature of the scenario considered and the efficient abstraction techniques introduced which yield compact finite state models. Additionally, it establishes insightful metrics for verification of the human driver model and synthesis of ADAS in driving situations, and paves the way for the establishment of a general framework to tackle similar driving scenarios.

\end{abstract}