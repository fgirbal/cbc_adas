\begin{abstract}
Research into safety in autonomous and semi-autonomous vehicles has, so far, largely been focused on testing and validation through simulation. Due to the fact that failure of these autonomous systems is potentially life-endangering, formal methods arise as a complementary approach.

This thesis studies the application of formal methods to the verification of different profiles of human drivers in a particular scenario (built using a cognitive architecture) and to the design of correct-by-construction Advanced Driver Assistance Systems (ADAS) using strategy synthesis.

The results show that the ADAS synthesised improve both safety and efficiency of the overall system when compared to the human driver alone. 

This dissertation models a complex scenario (a 2-lane highway and the interactions that arise with various profiles of drivers e.g. follow, crash or overtake another vehicle) and introduces efficient abstraction techniques to generate compact finite state models. Additionally, it establishes insightful metrics for verification and synthesis of ADAS in driving situations, and paves the way for the establishment of a general framework to tackle similar scenarios.
\end{abstract}