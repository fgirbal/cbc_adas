\chapter{Introduction}

Human safety is an extremely important aspect to consider when analysing possible transportation solutions. According to a study carried out by the National Highway Traffic Safety Administration in the USA, over $90\%$ of all road accidents are mainly attributed to errors of human drivers, caused by distractions, fatigue, alcoholic influence, among other factors \cite{crashes}. In an attempt to reduce these numbers, several car manufacturers have worked on solutions which minimise driver intervention through the introduction of autonomous features. Vehicles where these elements are present are ranked as autonomous vehicles of levels 2 or 3\footnote{Autonomous vehicles are ranked on six differing levels of autonomy, from 0 (not autonomous at all) to 5 (fully autonomous) \cite{wired}.}, and the features are generally described as Advanced Driver Assistance Systems (ADAS). Examples include Tesla's \textit{Autopilot} and Ford's \textit{Co-Pilot 360} \cite{autopilot1, ford}.

While ADAS are more practical for drivers and it would appear they are beneficial in terms of safety, there are two concerns regarding the way these systems are designed. Firstly, they are built by humans and, as such, the presence of software mistakes in the implementation is almost unavoidable, particularly due to the complexity of the systems in question. An example which supports this concern happened when, on March 23, 2018, a driver of a Tesla Model X died after the vehicle crashed into a highway divider in California with \textit{Autopilot} engaged \cite{autopilot2}. The accident was later attributed to a human error in \textit{Autopilot}. Secondly, most of the ADAS built lack the inclusion of the human cognitive process as a way to predict the actions drivers will take and adjust the suggestions/actions taken accordingly. 

In order to tackle these issues, formal verification - the act of proving or disproving the correctness of intended processes underlying a system with respect to a certain specification \cite{bk08} - is appropriate to guarantee requirements are fulfilled and the systems function correctly, and cognitive modelling - the approximation of the human cognitive processes using appropriate frameworks \cite{actr_1} - is useful for the accurate modelling of the human driver behaviour.

\section{Related Work}

There are several approaches to the study of safety in the context of autonomous and semi-autonomous vehicles. As Nidhi \textit{et al} pointed out in \cite{driving-to-safety}, a logical approach would be to test drive these vehicles in real world situations, create evaluation metrics, make observations on their performance, and apply statistical comparisons against the baseline that is the human driver. However, Nidhi \textit{et al} concluded that this would be infeasible, as the data required to make any statistically relevant conclusions on the safety of the vehicles would take tens, if not hundreds, of years to collect. An alternative solution is through the modelling and simulation of the autonomous vehicle, using frameworks which allow testing under different conditions, as proposed in \cite{sim1, sim2, sim3, sim4}. Several companies developing autonomous cars have turned to this solution as a way of obtaining vast amounts of data on the security and reliability of their vehicles, in order to continuously improve the systems they have designed. Despite this, it is imperative to recognise the shortcomings of simulation in safety evaluation of complex driver assistance systems which could have life-endangering impact \cite{challenges1, challenges2}. In this context, formal verification arises as a complementary approach to simulation, leading to automation and precision in the testing process.

The techniques used in formal verification range from model checking, the process of verifying through exhaustive search whether a model meets a given specification, to strategy synthesis, which corresponds to obtaining a the set of actions to be taken which is guaranteed to satisfy one or multiple requirements (if they exist) \cite{bk08, games2}. The strategies obtained through synthesis are, by definition, \textit{correct-by-construction}, in the sense that they are outputted in the process as a result of the fact that they are provably correct up to the level of representation of the model \cite{bk08, games2}. As such, they do not fall within the implementation mistakes that humans might. The process of strategy synthesis is widely used nowadays in software development \cite{software_1, software_2, software_3}. 

Due to the benefits that formal verification brings to the field, many researchers have used these tools to evaluate control systems in autonomous vehicles, to verify models of human driver behaviour, and study cooperation within fleets of these vehicles \cite{games, fv2, fv3, fv1}. In \cite{nilsson}, Nilsson \textit{et al.} took the first step towards correct-by-construction ADAS by synthesising the control software module for adaptive cruise control from formal specifications given in Linear Temporal Logic. In \cite{lam}, Lam modelled the behaviour of a distracted human driver using a cognitive architecture in a one-way street traffic light scenario (no other vehicles present) and studied how a driver assistance system could improve the safety of vehicle, following the work of \cite{salvucci_1, curzon}. Despite these advances, there are many open questions in this area, particularly whether real data validates the models constructed and, therefore, results obtained; how to deal with more complex environments and interactions without getting into representation problems such as state explosion (with an increase of the complexity of the system to be modelled, the number of variables required rises and the state space exponentially increases); and whether or not it is possible to represent an arbitrarily generic situation in this context.

In the case of safety in the context of semi-autonomous vehicles, the mentioned techniques are constrained by the accurate modelling of the human driver. The first integrated approaches to this problem, proposed in \cite{boer, older_1, older_3}, rely on modelling human behaviour through continuous controllers and specifically engineered frameworks. However, these methods lack the human behaviour variability attributed to the discrete nature of the control actions performed by the drivers. With this in mind, in \cite{salvucci_0} Salvucci \textit{et al} proposed a proof-of-concept of the introduction of human constrains in the driver model through the use of the cognitive architecture Adaptive Control of Thought-Rational (ACT-R) - a framework for specifying computational behavioural models of human cognitive performance, embodying both the abilities (e.g. memory storage and recall or perception) and constraints (e.g. limited motor performance or memory decay) of the human system \cite{salvucci_1}. In \cite{salvucci_1}, Salvucci proposed an updated version of the human driver model initially introduced in \cite{salvucci_0}, which was improved using the advances in ACT-R.

\section{Aims and Contributions}

Following some of the work presented in the previous section, the aim of this project is to study the application of such techniques to the verification of models of different profiles of human drivers (built using a cognitive architecture) and design correct-by-construction advanced driver assistance systems using strategy synthesis under multi-objective properties. 

This thesis extends the work of \cite{games, lam, salvucci_1} by considering a 2-lane highway scenario and the interactions that arise for various profiles of drivers (e.g. follow, crash into or overtake another vehicle), with the primary goal of decreasing accident risk through the synthesis of an advanced driver assistance system. It takes the first steps towards the ultimate goal of establishing a general framework to tackle the design of ADAS for different driving situations.

The scenario proposed distinguishes itself from previously studied ones due to the high complexity presented and the implementation of the human driver model using a cognitive architecture. Through this, the dissertation  introduces novel strategies for accurate abstraction of existing models in order to generate finite state space Discrete Time Markov Chain (DTMC) and Markov Decision Process (MDP) representations. This thesis is also focused on multi-objective synthesis for the advanced driver assistance system, with safety and time efficiency being optimised (e.g. to avoid situations where the system will always slow down instead of overtaking another vehicle), contributing to the practical usefulness of the system for the driver of the vehicle. 

\section{Thesis Outline}

The methodology of the dissertation is supported using several definitions and concepts from the literature which are presented in Chapter~\ref{sec:background}. Following this, Chapter~\ref{sec:human_driver} covers the methods applied to the construction of the human driver model, from the abstraction of Salvucci's integrated driver model to the establishment of properties to be verified on the model, as well as a graphical visualisation tool developed. In Chapter~\ref{sec:adas}, the driver model is transformed and augmented with a driver assistance system. Several designs are studied and compared, with the best performing one being the subject of the in-depth analysis presented in Chapter~\ref{sec:results}. In addition to the performance evaluation of the ADAS, Chapter~\ref{sec:results} is also composed of several examples of possible extensions of the scenario through the usage of different properties and assumptions. The thesis is concluded with an evaluation of the work developed and future directions that research in this area could follow.












